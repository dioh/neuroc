\documentclass[a4paper,spanish]{article}

\usepackage{url}
\usepackage{verbatim}

\input{preambulo}

\hyphenation{ob-te-ner}

\begin{document}

\materia{Herramientas computacionales en Neurociencias}
\titulo{Trabajo Pr\'{a}ctico}
\subtitulo{Pr\'actica 6 - Spike Sorting}
\abst{s }
\grupo{}
\claves{}
\integrante{Daniel J. Foguelman}{667/06}{dj.foguleman@gmail.com}


%\lhead{\parbox{7.4cm}{\small TP \textsc{Teor\'{i}a de Lenguajes}}}

\cfoot{$\thepage$ de \pageref{LastPage}}

\thispagestyle{empty}

\maketitle




%\tableofcontents

%\pagebreak

\section{Introducci'on}
\subsection{`? Por qu\'e Spike-Sorting?}

\subsection{Temas abarcados en la pr\'actica}

\subsection{Objetivos}

\section{Desarrollo}
\subsection{Datos, an\'alisis}
\subsection{Filtering}
\subsection{Feature an\'alisis}
\subsection{Clustering}


\section{La Pr\'actica}
\subsection{Problema 1}
En este problema se pide graficar un dataset que tiene la siguiente estructura:
\begin{tabular}{ c | c | c | c}
	spikes & cluster\_class & inspk & par \\
\end{tabular}

Y analizar la cantidad de neuronas que podr\'ian encontrarse en dichos datos.

Comenzamos trabajando sobre la variable spikes. Entendemos que estos datos ya se encuentran:
\begin{itemize}
	\item Filtrados
	\item Ya se ha hecho un trabajo de Spike detection (con Amplitude Threshold o alg\'un m\'etodo similar).  
\end{itemize}


El problema de Spike-Sorting, a grandes rasgos, trata de mapear un tren de spikes en m\'ultiples canales de medici\'on con el de una sola neurona. Las dificultades para distinguir cada neurona en particular puede deberse a:
\begin{itemize}
	\item Distintos canales miden la misma neurona desde distintos lugares y distancias, mostrando distintos valores. (Quin Quiroga cita Gold et al., 2006)
	\item La sincronizaci\'on inter-neuronal lleva a la superposici\'on de se\~nales. (Phase Synchronization).
	\item Mala elecci\'on de par\'ametros en los procesos de Detecci\'on de Spikes y de Features.
\end{itemize}

La pregunta que subyace de este an\'alisis ser\'ia entonces ¿cuantas neuronas podemos contar a partir de los datos que tenemos?
Sabemos que habr\'a al menos tantas neuronas como clases de features. ¿Pero cuantas m\'as habr\'a? ¿C\'omo distinguimos neuronas sincronizadas?

En la figura ~\ref{fig1} podemos observar cuatro features distintos, ¿cuantos otros features no son analizables a simple vista y precisan de un proceso m\'as sofisticado? 

Las preguntas que queremos responder utilizando alg\'un m\'etodo de Clustering entonces ser\'ian:
%%%% TODO: Reforzar preguntas FLOJO
\begin{itemize}
	\item Cada Feature corresponde al spike-train de\textbf{una} neurona o de varias? 
	\item Hemos distinguido \textbf{todos} los features? 
	\item Podemos encontrar se\~ales de otras neuronas en ellos?
\end{itemize}

Poniendolo en t\'erminos de data mining, podemos armar una partici\'on de las N observaciones en K clusters donde cada observaci\'on pertenece al cluster con media m\'as cercana.
Haciendo un an\'alisis meramente cuantitativo, podemos ver que para $N > K$, por el principio del palomar, dos observaciones pertenecer\'an al mismo cluster. Esto implica que por cada cluster, estaremos necesariamente encontrando se\~nales de varias neuronas, o bien sincronizadas, o bien que no pudimos separar en la etapa de filtrado o de detecci\'on de features.



%%%% TODO: Mejorar grafico.
\begin{figure}[htc]
	\centering
	\includegraphics[width=1.0\textwidth]{imgs/spikes_data_1.png}
	\caption{Spikes por tiempo. Datos de data\_sim1.}
	\label{fig1}
\end{figure}


%%%% Aca hablo de lo importante de clasificar

%%% Classification is the task of learning a target function f that maps each attribute set x to one of the predefined class labels y.  The target function is also known informally as a classification model.  A classification model is useful for the following purposes.



Veamos qu\'e sucede aplicando el m\'etodo de K-Means a los datos dados:

%%%% TODO: Mejorar grafico.
\begin{figure}[htc]
	\centering
	\includegraphics[width=1.0\textwidth]{imgs/clustering_4_data_1.png}
	\caption{Clustering en cuatro clases los datos en data\_sim1.}
	\label{fig2}
\end{figure}


%%%% Aca hablo de K-Means, propiedades, etc.
El algor\'itmo realiza una categorizaci\'on, en un espacio de categor\'ias fijo, agrupando las observaciones seg\'un su media. Cada observaci\'on pertenecer\'a a un cluster tal que se minimice las sumas de las medias de cada cluster.


%%%% Hablo de Confusion matrix
%%%% Misclassification error, como estimarlo, cuan malo es






\subsection{Problema 2}


\begin{figure}[htc]
	\centering
	\includegraphics[width=1.0\textwidth]{imgs/raw_cont_data.png}
	\caption{Datos crudos en cont\_data, dataset.}
	\label{fig3}
\end{figure}

\begin{verbatim}
>> graphic_wavelforms(a, kmeans(a, 8), 8);                           
>> graphic_wavelforms(a, kmeans(a, 10), 10);
>> a = amplitude_threshold(dat.data, dat.par.sr, std(dat.data)  * 2); 
\end{verbatim}

\end{document}
